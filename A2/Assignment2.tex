\documentclass[a4paper,12pt]{article}
\usepackage{graphicx, setspace, array, xcolor, mathtools, contour, tikz, amsthm, setspace, amsmath,amsfonts,amssymb, subcaption, fancyhdr, lipsum,multicol, geometry, gensymb}
\usepackage{colortbl}
\geometry{a4paper, margin=0.9in}
\contourlength{0.5pt}
\usetikzlibrary{patterns}
\usepackage[english]{babel}
\renewcommand{\qedsymbol}{$\blacksquare$}
\definecolor{darkgreen}{rgb}{0.0, 0.5, 0.0}
\pagestyle{fancy}
\fancyhf{}
\renewcommand{\headrulewidth}{0.4pt}
\fancyhead[L]{\rightmark}
\fancyhead[R]{\thepage}
\title{Modern Algebra}
\author{Assignment 2\\ \\ Yousef A. Abood\\ \\ ID: 900248250}
\date{September 2025}
\setlength{\parindent}{0pt}
\singlespacing
\parskip=1mm
\setcounter{section}{1}
\setcounter{subsection}{1}
\onehalfspacing
\begin{document}
\maketitle
\noindent\makebox[\linewidth]{\rule{15cm}{0.4pt}}
\section{Groups}
\subsection*{Problem 1}
\begin{itemize}
    \item [b)] Division of non-zero integers is not closed, as $\frac{1}{2} \notin \mathbb{Z}.$
    \item [d)] Multiplying $2 \times 2$ matrices with integer entries is closed. Pick $A,B,C 2\times 2$ marices, let $AB=C$ we know from the Linear Algebra course that $C$ has the same number of rows as $A$ and the same number of columns in $B$. Thus, $C$ is a $2 \times 2$ matrix.
\end{itemize}
\subsection*{Problem 2}
\begin{itemize}
    \item [a)] Subraction of integers is not associative, as $(5-3)-2=0\ne 5-(3-2)=4.$
    \item [b)] Division by non-zero rationals is not associative, as $(16/8)/4=\frac{1}{2}\ne 16/(8/4)=8.$
    \item [e)] Exponentiation of integers is not associative, as $(2^2)^3=64 \ne 2^{(2^3)}=2^8=256.$
\end{itemize}
\subsection*{Problem 3}
\begin{itemize}
    \item [c)] Let $g(x)=x^2, f(x)=x+1$, we observe that \[f(g(x))=x^2+1\ne g(f(x))=x^2+2x+1.\] So, $f(g(x))\ne g(f(x))$ and function composition of polynomials with real coefficients is not commutative.
    \item [d)] Let $A=\begin{bmatrix}
      1 & 2 \\
      3 & 4
    \end{bmatrix}, B=\begin{bmatrix}
      5 & 6 \\
      7 & 8
    \end{bmatrix},$ we observe that \[AB=\begin{bmatrix}
      19 & 22  \\
      43 & 50
    \end{bmatrix}, BA=\begin{bmatrix}
      23 & 34 \\
       31& 46
    \end{bmatrix}\] So, the multiplication of $2\times 2$ matrices with real entries is not commutative.
\end{itemize}
\subsection*{Problem 5}
\begin{itemize}
    \item [a)] The inverse of $13$ is $7$, as $(13+7)\mod 20 =0$
    \item [b)] The inverse of $13$ is $13$, as $(13*13) \mod 14=169 \mod 14=1.$
\end{itemize}
\subsection*{Problem 7}
\textbf{First reason:} The set is not closed under addition, as $3+5=8$ wich is an even number.\\
\textbf{Second reason:} It does not have the identity element.
\subsection*{Problem 14} $(ab)^3=(ab)(ab)(ab)$. Since multiplication is associative, we can remove the paranthases. So, $(ab)^3=(ab)(ab)(ab)=ababab.$\\
$(ab^{-2}c)^{-2}=(ab^{-2}c)^{-1}(ab^{-2}c)^{-1}$. By the \textit{Socks-Shoes Property} and the associativity of multiplication, $((ab^{-2})c)^{-1}=c^{-1}(ab^{-2})^{-1}=c^{-1}(b^{-2})^{-1}a^{-1}=c^{-1}b^{2}a^{-1}.$ Thus, \[(ab^{-2}c)^{-1}(ab^{-2}c)^{-1}= c^{-1}b^{2}a^{-1}c^{-1}b^{2}a^{-1}.\]
\subsection*{Problem 16} 
\begin{proof}
    We construct the cayley table of the operation:
\[\begin{tabular}{|c|c|c|c|c|c|}
    \hline
    &5&15&25&35\\
    \hline
    5&25&35&5&15\\
    15&35&25&15&5\\
    25&5&15&25&35\\
    35&15&5&35&25\\
    \hline
    
\end{tabular}\]
From the table, we observe that the set is closed under multiplication modulo 40, $25$ is the identity element, and every element has an inverse which is the number itself. The modular multiplication is associative by definition. 
Therefore, the set $\{5,15,25,35\}$ with the operation multiplication modulo 40 is a group.
\end{proof}
We know that $U(8)=\{1,3,5,7\}.$
We construct the cayley table of $U(8)$:
\[\begin{tabular}{|c|c|c|c|c|c|}
    \hline
    &1&3&5&7\\
    \hline
    1&1&3&5&7\\
    3&3&1&7&5\\
    5&5&7&1&3\\
    7&7&5&3&1\\
    \hline
\end{tabular}\]
We see that the group is similar to $U(8),$ they both have four elements and each element is the inverse of itself.
\subsection*{Problem 18}
From the cayley table of $D_4$, we can conclude that:
\[K=\{R_0, R_{180}\}, L=\{R_0, R_{180}, H, V, D, L\}\] 
\subsection*{Problem 33}
\[\begin{tabular}{|c|c|c|c|c|c|c|}
    \hline
     &e&a&b&c&d\\
    \hline
    e&e&a&b&c&d\\
    a&a&b&c&d&e\\
    b&b&c&d&e&a\\
    c&c&d&e&a&b\\
    d&d&e&a&b&c\\
    \hline
\end{tabular}\]
\subsection*{Problem 34}
\textbf{First part:}
\begin{proof}
    We need to proof that $(ab)^2=a^2b^2$ iff $ab=ba.$ We start by the forward direction. Assume $(ab)^2=a^2b^2$ is true, so we have that $(ab)(ab)=abab=a^2b^2.$ We multiply each side by $b^{-1}$ from the right to get \begin{align*}aba(bb^{-1})&=a^2b^2b^{-1}\\aba(e)&=a^2b(bb^{-1})\\aba&=a^2be=a^2b\end{align*}
    Then, we multiply each side by $a^{-1}$ from the left:
    \begin{align*}
        a^{-1}aba &=a^{-1}a^2b\\
        (a^{-1}a)ba &=a^{-1}aab\\
        eba &= (a^{-1}a)ab\\
        ba &= eab\\
        ba &= ab.
    \end{align*}
    Hence, we proved that $ab=ba$ if $(ab)^2=a^2b^2$.\\
    For the backward direction, we assume that $ab=ba$. Then, $(ab)^2=(ab)(ab)=abab=(ab)ab=((ab)a)b=(a(ba))b=(a(ab))b=((aa)b)b=(a^2b)b=a^2bb=a^2b^2.$ Thus, we proved that if $ab=ba$ then $(ab)^2=a^2b^2.$ Therefore, we proved that $(ab)^2=a^2b^2$ iff $ab=ba.$
\end{proof}
\textbf{Second part:}
\begin{proof}
    We need to proof that $(ab)^{-2}=b^{-2}a^{-2}$ iff $ab=ba.$ We start by the forward direction. Assume $(ab)^{-2}=b^{-2}a^{-2}$ is true, so we have that $(ab)^{-1}(ab)^{-1}=(b^{-1}a^{-1})(b^{-1}a^{-1})=b^{-1}a^{-1}b^{-1}a^{-1}=b^{-2}a^{-2}.$ We multiply each side by $a$ from right to get 
    \begin{align*}b^{-1}a^{-1}b^{-1}a^{-1}a&=b^{-2}a^{-2}a\\b^{-1}a^{-1}b^{-1}a^{-1}a&=b^{-2}a^{-1}a^{-1}a\\b^{-1}a^{-1}b^{-1}(a^{-1}a)&=b^{-2}a^{-1}(a^{-1}a)\\b^{-1}a^{-1}b^{-1}e&=b^{-2}a^{-1}e\\b^{-1}a^{-1}b^{-1}&=b^{-2}a^{-1}
    \end{align*}
    Then, we multiply each side by $b$ from the left:
    \begin{align*}
        bb^{-1}a^{-1}b^{-1}&=bb^{-2}a^{-1}\\
        (bb^{-1})a^{-1}b^{-1}&=bb^{-1}b^{-1}a^{-1}\\
        (bb^{-1})a^{-1}b^{-1}&=(bb^{-1})b^{-1}a^{-1}\\
        ea^{-1}b^{-1}&=eb^{-1}a^{-1}\\
        a^{-1}b^{-1}&=b^{-1}a^{-1}\\
    \end{align*}
    We take the inverse of both sides:
    \begin{align*}
        (a^{-1}b^{-1})^{-1}&=(b^{-1}a^{-1})^{-1}\\
        ba&=ab.
    \end{align*}
    Hence, we proved that $ab=ba$ if $(ab)^{-2}=b^{-2}a^{-2}$.\\
    For the backward direction, we assume that $ab=ba$. Then, $(ab)^{-2}=(ab)^{-1}(ab){-1}=b^{-1}a^{-1}b^{-1}a^{-1}=(b^{-1}a^{-1})b^{-1}a^{-1}=((b^{-1}a^{-1})b^{-1})a^{-1}=(b^{-1}(a^{-1}b^{-1}))a^{-1}=(b^{-1}(b^{-1}a^{-1}))a^{-1}=((b^{-1}b^{-1})a^{-1})a^{-1}=(b^{-2}a^{-1})a^{-1}=b^{-2}a^{-1}a^{-1}=b^{-2}a^{-2}.$ Thus, we proved that if $ab=ba$ then $(ab)^{-2}=b^{-2}a^{-2}.$ Therefore, we proved that $(ab)^{-2}=b^{-2}a^{-2}$ iff $ab=ba.$
\end{proof}
\subsection*{Problem 47}
\begin{proof}
    Suppose $G$ is a group with the property that the square of every element is the identity, then every element is the inverse of itself. We want to prove that the group is \textit{Abelian.} Choose $a,b \in G$, we observe that:
    \begin{align*}
        (ab)^2&=e \\
        (ab)(ab)&=e\\
        abab&=e\\
        ababb^{-1}&=eb^{-1}\\
        aba(bb^{-1})&=b^{-1}\\
        abae&=b^{-1}\\
        aba&=b^{-1}\\
        a^{-1}aba&=a^{-1}b^{-1}\\
        (a^{-1}a)ba&=a^{-1}b^{-1}\\
        eba&=a^{-1}b^{-1} \\
         ba&=a^{-1}b^{-1}\\
    \end{align*}
    By our hypothesis, we know that $b^{-1}=b, a^{-1}=a$. Thus, $ba=ab$. Therefore, we proved that if the group with the property that the square of every element is the identity, then the group is Abelian.
\end{proof}
\end{document}