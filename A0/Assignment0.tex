\documentclass[a4paper,12pt]{article}
\usepackage{graphicx, setspace, array, xcolor, mathtools, contour, tikz, amsthm, setspace, amsmath,amsfonts,amssymb, subcaption, fancyhdr, lipsum,multicol, geometry}
\usepackage{colortbl}
\geometry{a4paper, margin=1in}
\contourlength{0.5pt}
\usetikzlibrary{patterns}
\usepackage[english]{babel}
\renewcommand{\qedsymbol}{$\blacksquare$}
\definecolor{darkgreen}{rgb}{0.0, 0.5, 0.0}
\pagestyle{fancy}
\fancyhf{}
\renewcommand{\headrulewidth}{0.4pt}
\fancyhead[L]{\rightmark}
\fancyhead[R]{\thepage}
\title{Modern Algebra}
\author{Assignment 0\\ \\ Yousef A. Abood\\ \\ ID: 900248250}
\date{September 2025}
\setlength{\parindent}{0pt}
\singlespacing
\parskip=1mm
\setcounter{section}{-1}
\setcounter{subsection}{1}
\onehalfspacing
\begin{document}
\maketitle
\noindent\makebox[\linewidth]{\rule{15cm}{0.4pt}}
\section{Parliments}
\subsection*{Problem 2}
\begin{itemize}
    \item [d)] The divisors of 21 are: $21,7,3,1$. The divisors of 50 are: $50, 25, 10, 5, 2, 1$. So the $gcd(21,50)=1.$\\ The $lcm(21, 50)=1050.$
\end{itemize}
\subsection*{Problem 4} \begin{proof}
We pick $s,t \in \mathbb{Z}.$ For the sake of contradiction, assume $s,t$ are unique. That means we can only find one value for $s$ and one value for $t$ such that they satisfy the equation $1=7s+11t.$ We choose $s=-3, t=2$, so $7\times -3+11 \times 2=-21+22=1,$ which satisfies the equation. Choose $s=8, t=-5,$ we see that $7\times 8+11 \times -5=56+(-55)=1,$ which satisfies the equation. We see we found two values for $s,t$ each that satisfies the equation. Therefore, $s$ and $t$ are not unique.
\end{proof}
\subsection*{Problem 7} \begin{proof}
    We pick $a,b,n \in \mathbb{Z}.$ For the forward direction, we assume that $a$ $ mod$ $n = b$ $mod$ $n$. Since we can write $a=q_1n+r_1$ and $b=q_2n+r_2$ by the division algorithm, and $r_1=r_2$ by our assumption. Then $a-b=q_1n-q_2n=n(q_1-q_2)$ and $n\mid a-b.$ For the backward direction, we assume that $n\mid a-b,$ so there is a $k \in \mathbb{Z}$ such that $nk=a-b.$ Then we use the division algorithm to divide $a,b$ by $n$. So we get $a=q_1n+r_1, b=q_2n+r_2$, where $0\leq r_1 < n, 0 \leq r_2 <n.$ To show that $a mod n = b mod n$ we need to show that $r_1=r_2.$ WLOG, we assume $r_1\geq r_2$. By our assumption, we know that $n\mid a-b$ and \[a-b=(q_1n+r_1)-(q_2n+r_2)=(q_1n-q_2n)+(r_1-r_2)=nk.\]
Since $0\leq r_1 < n, 0 \leq r_2 <n.$ and $r_1\geq r_2$, then $0 \leq r_1-r_2<m$. Now, we have that \[nk=(q_1n-q_2n)+(r_1-r_2)\iff (r_1-r_2)=n(q_1-q_2-k)\], so $m\mid r_1-r_2.$ But we know that $0 \leq r_1-r_2<m.$ Hence, $r_1-r_2$ must be zero and $r_1=r_2.$ Therefore, that satisfies the proof.
\end{proof}
\subsection*{Problem 10}
\begin{proof}
   Let $a,b \in \mathbb{Z^+}$, and $d= gcd(a,b), m=lcm(a,b)$. For the first part, We can apply prime factorization to booth $a,b,t$ to get \begin{align*}a=p_1^{f_1} \cdot p_2^{f_2}\cdot p_3^{f_3}\cdot \cdot \cdot p_n^{f_n}\\ b=p_1^{g_1}\cdot p_2^{g_2}\cdot p_3^{g_3}\cdot \cdot \cdot p_n^{g_n}\\ t=p_1^{r_1}\cdot p_2^{r_2}\cdot p_3^{r_3}\cdot \cdot \cdot p_n^{r_n}\end{align*} Where $f,g,r \in \mathbb{Z}.$ By theorem 5.4.5. in the discrete math lecture notes by Dr. Daoud Siniora, we can write \[d=p_1^{min(f_1, g_1)}\cdot p_2^{min(f_2, g_2)}\cdot p_3^{min(f_3, g_3)}\cdot \cdot \cdot p_n^{min(f_n,g_n)}\].
Since we know that $t$ divides $a$, then for every $p$ in the prime factorization of $t$, the exponents must be less than or equal to the exponents of $p$ in the prime factorization of $a$. So for all $i=1,....,k$, $r_i\leq f_i$.
Since we know that $t$ divides $b$, then for every $p$ in the prime factorization of $t$, the exponents must be less than or equal to the exponents of $p$ in the prime factorization of $b$. So for all $i=1,....,k$, $r_i\leq g_i$. From the previous steps, for all $r_i<=min(g_i, f_i).$ Hence, $t$ divides $d.$
\\For the next part, We can apply prime factorization to booth $a,b,s$ to get \begin{align*}a=p_1^{f_1}\cdot p_2^{f_2}\cdot p_3^{f_3}\cdot \cdot \cdot p_n^{f_n}\\ b=p_1^{g_1}\cdot p_2^{g_2}\cdot p_3^{g_3}\cdot \cdot \cdot p_n^{g_n}\\ s=p_1^{e_1}\cdot p_2^{e_2} \cdot p_3^{e_3} \cdot \cdot \cdot p_n^{e_n}\end{align*} Where $f,g,e \in \mathbb{Z}.$ By theorem 5.4.5. in the discrete math lecture notes by Dr. Daoud Siniora, we can write \[m=p_1^{max(f_1, g_1)}\cdot p_2^{max(f_2, g_2)}\cdot p_3^{max(f_3, g_3)}\cdot \cdot \cdot p_n^{max(f_n,g_n)}\].
Since we know that $s$ is a multiple of $a$, then for every $p$ in the prime factorization of $s$, the exponents must be greater than or equal to the exponents of $p$ in the prime factorization of $a$. So for all $i=1,....,k$, $e_i\geq f_i$.
Since we know that $s$ is a multiple of $b$, then for every $p$ in the prime factorization of $s$, the exponents must be greater than or equal to the exponents of $p$ in the prime factorization of $b$. So for all $i=1,....,k$, $e_i\geq g_i$. From the previous steps, for all $e_i\geq max(g_i, f_i).$ Hence, $s$ is a multiple of $m.$ 
\end{proof}
\subsection*{Problem 11}
\begin{proof}
Let $a,n \in \mathbb{Z^+}$ and $d=gdc(a,n).$ For the forward direction, Assume the equation $ax$ $mod$ $n=1.$ has a solution. Using the division algorithm, we can write the equation as $ax=qn+1 \implies ax-qn=1.$ Since $d$ divides $a,n$, it divides any linear combination of $a,n$. Then, we see that $d\mid ax-qn, d\mid1$. Since $d=gcd(a,n)$, then it is a positive integer. We know that the only positive integer that divides $1$ is $1$. Hence, $d$ must equal $1$. For the backward direction, we assume $d=1$. Since $d=gcd(a,n)$, we can write it as a linear combination of $a,n$. So $1=ca+tn$, where $c,t\in \mathbb{Z}.$ We observe that $ca+tn=ax-qn$, we can take $x=c, q=-t.$ Hence, the equation has a solution.    
\end{proof}
\subsection*{Problem 13}
\begin{proof}
    Let $m,n,r \in \mathbb{Z}$. Suppose $m,n$ are coprimes, so $gcd(m,n)=1$. Since we know that the $gcd(m,n)$ is a linear combination of $m,n$, we can write that $1=cm+tn$, where $c,t \in \mathbb{Z}.$ We observe that by multiplying both sides by $r$, we get that $r=rcm+rtn=(rc)m+(rt)n$.
Since, $r,t,c$ are integers, then $rt, rc$ are integers. Let $rt=y, rc=x$, so $r=mx+ny.$ Therefore, that satisfies the proof.
\end{proof}
\subsection*{Problem 20}
\begin{proof}
    Let $p_1,p_2,....,p_n$ be primes. For the sake of contradiction, Assume that $p_1p_2\cdot \cdot \cdot p_n+1$ is divisible by one of these primes .(1) Pick a prime $p_i$, we know that $p_i\mid p_1p_2\cdot \cdot \cdot p_i \cdot \cdot \cdot p_n .(2) $ By (1) and (2), $p_i\mid p_1p_2\cdot \cdot \cdot p_n-(p_1p_2\cdot \cdot \cdot p_n+1)$. And $(p_1p_2\cdot \cdot \cdot p_n)-(p_1p_2\cdot \cdot \cdot p_n)-1=-1,$. But $-1$ does not have a prime divisor, so $p_i\nmid p_1p_2\cdot \cdot \cdot p_n-(p_1p_2\cdot \cdot \cdot p_n+1)$, which is a contradiction. Therefore, we proved that $p_1p_2\cdot \cdot \cdot p_n+1$ is not divisible by any prime.
\end{proof}
\subsection*{Problem 28}
\begin{proof}
  We proceed by mathematical induction. Let $P(n)=: 2^n3^{2n}-1$ is divisible by $17.$\\
\textbf{Base case:} We show the statement $P(n)$ is true for $P(0).$ When $n=0$, $2^03^0-1=1-1=0,$ clearly divisible by $17$.\\
\textbf{Induction step:} We need to show that $P(n) \to P(n+1)$ for all $n\geq 1$, where $n \in \mathbb{Z^+}$ Suppose $P(n)$ is true, we need to show that $P(n+1)$ is true as well. We observe that \[2^{n+1}3^{2n+2}-1=2\cdot 9\cdot (2^n\cdot 3^{2n})-1\]. By our assumption, $2^n3^{2n}-1=17k \implies 2^n3^{2n}=17k+1,$ where $k \in \mathbb{Z}.$ We subistitue back in
\begin{align*}
    2\cdot 9\cdot (2^n\cdot 3^{2n})-1 &\overset{IH}{=} 2\cdot 9\cdot (17k+1)-1\\
    &= 18(17k+1)-1=18\cdot 17 k+ 18-1\\
    &=17 \cdot 18 k -17=17(18k-1).
\end{align*}
Hence, we proved that $2^{n+1}3^{2n+2}-1$ is divisible by $17$ and $P(n+1)$ is true. Therefore, We proved that $2^n3^{2n}-1$ is divisible by $17$ for all $n \in \mathbb{Z^+}$  
\end{proof}
\subsection*{Problem 33}
\begin{proof}
    By definition, we can write the mathematical induction with predicate logic as:
\[(P(0)\wedge \forall n(P(n)\to P(n+1)))\leftrightarrow \forall nP(n)\]
To proof the forward direction, we assume that $(P(0)\wedge \forall n(P(n)\to P(n+1)))$ is true. We need to show that $\forall n P(n)$ is true. For the sake of the contradiction, suppose that $\forall n P(n)$ is not true. That mean we have some $k \in \mathbb{N}$ such that $P(k)$ does not hold. We construct the set $S=\{n\in \mathbb{N}\mid \neg P(n)\}$, which contains all the elements that does not satisfy $P(n)$. By our assumption, the set $S$ is not empty.
By the \textbf{The well-ordering principle}, we deduce that $S$ has a least element, which we call $t$. Since $t$ is the first element that does not satisfy $P(n),$ then $P(t-1)$ holds. Moreover, $t \neq 0$, as $t$ does not have the property $P$. So, $t \geq 1$ and $t-1\geq 0,$ such that $t-1$ is a natural number such that $P(t-1)$ holds. But, by our assumption, we have $P(t-1) \to P(t).$ Since we have that $P(t-1) \to P(t)$ and $P(t-1)$ are true, then $P(t)$ is true, which contradicts that $P(t)$ is false. Therefore, $\forall n P(n)$ is true, where $n \in \mathbb{N}.$\\
For the backward direction, we assume that $\forall n P(n)$ is true, which implies $P(0)$ is true, and $P(n)\to P(n+1)$ is true. 
\end{proof}
\subsection*{Problem 35}
\begin{proof}
    We proceed by mathematical induction. Let $P(n)=: n^3+(n+1)^3+(n+2)^3$ is a multiple of 9. for all $n \in \mathbb{Z^+}.$\\
\textbf{Base case:} We need to show that $P(1)$ is true. $P(1)=1+8+27=36$, which is clearly a multiple of $9.$\\
\textbf{Induction step:} We need to show that $P(n) \to P(n+1)$ for all $n \geq 1.$ Suppose $P(n)$ is correct, we need to show that $P(n+1)$ is true as well. We observe by $P(n)$, then $n^3+(n+1)^3+(n+2)^3=9k$, where $k \in \mathbb{Z}.$
Then, 
\begin{align*}
    (n+1)^3+(n+2)^3+(n+3)^3&=(n+1)^3+(n+2)^3+n^3+9n^2+27n+27\\
    &=n^3+(n+1)^3+(n+2)^3+9n^2+27n+27\\
&\overset{IH}{=}9k+9n^2+27n+27\\
    &=9(k+n^2+3n+3).
\end{align*}
Which is a multiple of $9$. Therefore, we proved that for all positive integers $n^3+(n+1)^3+(n+2)^3$ is a multiple of 9.
\end{proof}
\subsection*{Problem 57}
\subsection*{Problem 58}
\subsection*{Problem 59}
\subsection*{Problem 63}

\end{document}