\documentclass[a4paper,12pt]{article}
\usepackage{graphicx, setspace, array, xcolor, mathtools, contour, tikz, amsthm, setspace, amsmath,amsfonts,amssymb, subcaption, fancyhdr, lipsum,multicol, geometry, gensymb}
\usepackage{colortbl}
\geometry{a4paper, margin=1in}
\contourlength{0.5pt}
\usetikzlibrary{patterns}
\usepackage[english]{babel}
\renewcommand{\qedsymbol}{$\blacksquare$}
\definecolor{darkgreen}{rgb}{0.0, 0.5, 0.0}
\pagestyle{fancy}
\fancyhf{}
\renewcommand{\headrulewidth}{0.4pt}
\fancyhead[L]{\rightmark}
\fancyhead[R]{\thepage}
\title{Modern Algebra}
\author{Assignment 1\\ \\ Yousef A. Abood\\ \\ ID: 900248250}
\date{September 2025}
\setlength{\parindent}{0pt}
\singlespacing
\parskip=1mm
\setcounter{section}{0}
\setcounter{subsection}{1}
\onehalfspacing
\begin{document}
\maketitle
\noindent\makebox[\linewidth]{\rule{15cm}{0.4pt}}
\section{Dihedral Groups}
\subsection*{Problem 2}
\begin{tabular}{|c|c|c|c|c|c|c|}
    \hline
    &$e$&$R_{60^\circ}$&$R_{120^\circ}$&$s_1$&$s_2$&$s_3$\\ \hline
    $e$&$e$&$R_{60^\circ}$&$R_{120^\circ}$&$s_1$&$s_2$&$s_3$\\ \hline
    $R_{60^\circ}$&$R_{60^\circ}$&$R_{120^\circ}$&$e$&$s_3$&$s_1$&$s_2$\\ \hline
    $R_{120^\circ}$&$R_{120^\circ}$&$e$&$R_{60^\circ}$&$s_2$&$s_3$&$s_1$\\ \hline
    $s_1$&$s_1$&$s_2$&$s_3$&$e$&$R_{60^\circ}$&$R_{120^\circ}$\\ \hline
    $s_2$&$s_2$&$s_3$&$s_1$&$R_{120^\circ}$&$e$&$R_{60^\circ}$\\ \hline
    $s_3$&$s_3$&$s_1$&$s_2$&$R_{60^\circ}$&$R_{120^\circ}$&$e$ \\ \hline
\end{tabular}\\ \\
$D_3$ is not Abelien, a counter example for commutativity is $R_{120^\circ} \cdot s_1=s_3$ but $ s_1\cdot R_{120^\circ}=s_2$
\subsection*{Problem 3}
\begin{itemize}
    \item [a)] $\{V\}$
    \item [b)] $\{R_{270^\circ}\}$
    \item [c)] $\{R_{0^\circ}\}$
    \item [d)] $\{R_{0^\circ},R_{180^\circ}, H,V,D,L\}$
    \item [e)] $\emptyset$
\end{itemize}
\subsection*{Problem 5}
For an n-gon, we have $2n$ of operations that preserve symmetry, these operations are rotations and reflections.\\
\textbf{For rotations}, there are $n$ possible rotations for every n-gon including the identity element. The rotations other than the identity are the multiples of $\frac{360}{n}$ degrees. We can represent the rotations by $r^c$, where $1\le c\le n-1$ is the number we can multiply $\frac{360}{n}$ with. Thus, the available rotations for an n-gon are: $e,r^1, r^2, r^3,...,r^{n-1}$.\\
\textbf{For reflections}, the available reflections differ whether $n$ is even or odd.
\begin{itemize}
    \item []\textbf{\textit{n} is odd:}
    The available reflections are the reflections about the axes of symmetry, which are axes from one vertix to the midpoint of the opposite side. Then, there are $n$ reflections as there are $n$ verticies.
    \item []\textbf{\textit{n} is even:}
    The available reflections are the reflections about the axes of symmetry, $\frac{n}{2}$ of them are axes from one vertix to the opposite vertix,and the other $\frac{n}{2}$ are those from one midpoint of one side to the midpoint of the opposite side. Then there are $n$ reflections.
\end{itemize}
\subsection*{Problem 13}
We observe the cayley table of $D_4$ in Modern Algebra Lecture Notes by Dr. Daoud Siniora. We find that $VR_{90^\circ} = R_{90^\circ}H$, but $V \ne H.$
\subsection*{Problem 21}
If we applied any reflection twice, we get the identity element, so $X$ cannot be a reflection. Since $Y \ne R_{90}$, then it can be $e,R_{180}, R_{270}$.It cannot be reflection as reflection after rotation leads to a reflection. By observing the same table in the Lecture Notes that we mentioned in Problem 13, we see that $R_{270}R_{180}=R_{180}R_{270}=R_{90}e=R_{90}$. But $R_{270},R_{90}$ cannot be represented in $X^2$ as $R_{270}=R_{90}R_{90}R_{90}.$ and $R_{90}$ is the smallest rotation we can do. Thus, $X=R_{90}$ and $X^2Y=R_{180}R_{270}=R_{90}R_{90}R_{270}=R_{90}$,and $Y=R_{270} \ne R_{90}.$
\end{document}