\documentclass[a4paper,12pt]{article}
\usepackage{graphicx, setspace, array, xcolor, mathtools, contour, tikz, amsthm, setspace, amsmath,amsfonts,amssymb, subcaption, fancyhdr, lipsum,multicol, geometry, gensymb}
\usepackage{colortbl}
\geometry{a4paper, margin=0.9in}
\contourlength{0.5pt}
\usetikzlibrary{patterns}
\usepackage[english]{babel}
\renewcommand{\qedsymbol}{$\blacksquare$}
\definecolor{darkgreen}{rgb}{0.0, 0.5, 0.0}
\pagestyle{fancy}
\fancyhf{}
\renewcommand{\headrulewidth}{0.4pt}
\fancyhead[L]{\rightmark}
\fancyhead[R]{\thepage}
\title{Modern Algebra}
\author{Assignment 3\\ \\ Yousef A. Abood\\ \\ ID: 900248250}
\date{September 2025}
\setlength{\parindent}{0pt}
\singlespacing
\parskip=1mm
\setcounter{section}{2}
\setcounter{subsection}{1}
\onehalfspacing
\begin{document}
\maketitle
\noindent\makebox[\linewidth]{\rule{15cm}{0.4pt}}
\section{Subgroups}
\subsection*{Problem 2}
$\langle \frac{1}{2} \rangle$ in $\textbf{Q}=\{\cdot\cdot\cdot, \frac{-3}{2}, -1, \frac{-1}{2}, 0, \frac{1}{2},1,\frac{3}{2}, \cdot\cdot\cdot\}$\\
$\langle \frac{1}{2} \rangle$ in $\textbf{Q}^*=\{\cdot\cdot\cdot, \frac{1}{8},\frac{1}{4},\frac{1}{2},1,2,4,8, \cdot\cdot\cdot\}$
\subsection*{Problem 4}
\begin{proof}
    Suppose we have a group $G$. We pick $g \in G$ and suppose its order is $n\in \mathbb{Z}.$ To prove that an element $t$ has order $n$ we need to prove that $t^n=e$ and there is no integer $s<n$ which satisfies $t^s=e.$ We know that $g^n=e.$ Observe that $(g^{-1})^n=(g^n)^{-1}=e^{-1}.$But the inverse of the identity element is the identity element. Thus, $(g^{-1})^n=e.$ For the second part, assume that we have $s \in \mathbb{Z}$, and $(g^{-1})^s=e.$ We know that $(g^{-1})^s=(g^s)^{-1}=e.$ Now, multiply both sides by $g^s$ to get \begin{align*}(g^s)^{-1} g^s&=eg^s\\ e&=g^s. \end{align*} 
    We know that $n$ is the least integer that satisfies $g^n=e.$ So, $s>=n,$ which contradicts our assumption that $s<n.$ So, $s$ Therefore, we proved that for any group, any element and its inverse have the same order.
\end{proof}
\subsection*{Problem 6}
\begin{itemize}
    \item [b)] $|a|=4, |b|=3, |a+b|=12.$
\end{itemize}
\subsection*{Problem 7}
\begin{align*}
    (a^4c^{-2}b^4)^{-1}=(a^{6-2}c^{-2}b^{7-3})^{-1}&=((a^6a^{-2})c^{-2}(b^7b^{-3}))^{-1}\\
    &=((ea^{-2})c^{-2}(eb^{-3}))^{-1}\\
    &=(a^{-2}c^{-2}b^{-3})^{-1}\\
    &=(a^{-2}(c^{-2}b^{-3}))^{-1}\\
    &=((c^{-2}b^{-3})^{-1}(a^{-2})^{-1}) \text{     Using \textit{Socks-Shoes Property}}\\
    &=(b^{3}c^{2}a^2) \text{     Using \textit{Socks-Shoes Property}}
\end{align*}
\subsection*{Problem 10}
We observe the cayley table of $D_4$ and get that:\\
$\{R_{0},R_{90},R_{180},R_{270}\}, \{R_{180},R_{0},H,V\}, \{R_{180},R_{0},D,L\}$ are the possible subgroups from $D_4.$
\subsection*{Problem 19}
\begin{proof}
    Let $a$ is a group element which has an infinite order. That implies that there is no $s \in \mathbb{Z^+}$ that satisfies $a^n=e.$ We pick $n,m \in \mathbb{Z}.$ We want to proof that if $m \ne n$ then $a^m \ne a^n.$ Assume, $m \ne n,$ we need to show $a^m \ne a^n.$ For the sake of contradiction, assume that $a^m=a^n.$ Without loss of generality, assume $m>n.$ Then, we multiply both sides by $(a^n)^{-1}:$ 
    \begin{align*}
        a^m(a^n)^{-1}=a^n (a^n)^{-1}\\
        a^m a^{-n}=e\\
        a^{m-n}=e.
    \end{align*}
    We see that we found an integer $m-n$ such that $g^{m-n}=e.$ But we know that if an element $g$ has infinite order then there is no integer $s$ such that $g^s=e.$ So we see that we clearly reached a contradiction. So, $a^m \ne a^n$ must be true. Therefore, we proved that $a^m \ne a^n$ when $m \ne n$ for every group element with infinite order.
\end{proof}
\subsection*{Problem 30}
$H$ must be the group of even integers.
\begin{proof}
We see that the group is proper, so it cannot be the group of integers. We know that the group has the elements $18, 30, 40$, and so it must has their multiples. Since these elements are in the group, we can get any linear combination of them using addition. That is \begin{center}
    $(18x+30y+40z)$, where $x,y,z \in \mathbb{Z}.$
\end{center}
By \textit{Bezout's identity}, any linear combination of numbers is a multiple of their \textit{gcd.} Thus, the group these elements create is the group that contains multiples of their gcd, which is \[\langle gcd(18,30,40)\rangle=\langle 2 \rangle.\]
Therefore, we can see $H$ is clearly the group of all even integers.
\end{proof}
\subsection*{Problem 34}
\begin{proof}
    Since $H,K$ are subgroups of $G$, then every element in both $H,K$ is an element in $G.$ By definition, we know that the intersection between two sets is the shared elements among these sets. We pick $s,t \in K \cap H.$ We know that $s,t \in H$, then $st \in H$. We also know that $s,t \in K$, then $st \in K$. Thus $st \in H \cap K$ and the operation is closed. Since $H,K$ are groups, they both have the identity element. So, it is clear that $e \in H \cap K.$ Since, $a \in H$ and $a \in K$ then $a^{-1}\in H, a^{-1}\in K.$ Therefore, we showed that $H \cap K$ is a subgroup of $G.$
\end{proof}
\textbf{For the second part,} the same proof extend to any number of subgroups.
\end{document}